This program might be use to compare the execution time of two algorithms from different complexity classes to solve the filter problem.

Based on some preprocessing directives setup while compiling this program, it is possible to print out the execution time of the chosen algorithm for a given input of size defined by command line argument.

\section*{2. The Problem}

The filter problem is\+: \begin{quote}
Given the range {\ttfamily \mbox{[}first;last)} defined over an array of integers, remove all the null and negative elements, and return a pointer to the new {\ttfamily last}, which should define a range containing the filtered elements. The processes must preserve the relative order of the elements that will be kept in the new range. \end{quote}


\subsection*{Example}

Given the range {\ttfamily \mbox{[}first;last)} with the following values \begin{quote}
-\/3, 4, 8, 0, -\/2, 7, 0, 12, -\/1 \end{quote}


the program should return the pointer {\ttfamily last} so that if we print {\ttfamily \mbox{[}first;last)} again we get \begin{quote}
4, 8, 7, 12 \end{quote}


If all elements in the input range are null or negative, the output range {\ttfamily \mbox{[}first, last)} should be empty.

If all elements in the input range are positive, the output range {\ttfamily \mbox{[}first,last)} should be the same as the input.

\section*{3. Compiling}

To compile you may enter this at the terminal prompt\+:


\begin{DoxyCode}{0}
\DoxyCodeLine{g++ -Wall -std=c++11 filter\_runtime.cpp -o filter}
\end{DoxyCode}


Alternatively, a cmake file is available. Follow the steps below\+:


\begin{DoxyEnumerate}
\item Create a build directory\+: {\ttfamily mkdir build}.
\item Change into the build directory\+: {\ttfamily cd build}.
\item Generate the project file (Makefile)\+: {\ttfamily cmake -\/G \char`\"{}\+Unix Makefiles\char`\"{} ..}
\item Compile each target ({\ttfamily timing\+\_\+filter} and {\ttfamily run\+\_\+tests}) or both targets at the same time\+: {\ttfamily cmake -\/-\/build . -\/-\/configure Release -\/-\/target timing\+\_\+filter} or {\ttfamily cmake -\/-\/build . -\/-\/configure Release -\/-\/target run\+\_\+tests}
\end{DoxyEnumerate}

\subsection*{Setting the execution options}

The program has several {\itshape preprocessing directives} that might be added to the compile line and enables you to choose between the following running options\+:


\begin{DoxyItemize}
\item We may choose between a {\itshape linear} and a {\itshape quadractic} algorithm to solve the problem.
\begin{DoxyItemize}
\item Add {\ttfamily -\/D A\+L\+GO=\char`\"{}\+Q\+U\+A\+D\char`\"{}} to choose the quadratic algorithm.
\item Add {\ttfamily -\/D A\+L\+GO=\char`\"{}\+L\+I\+N\char`\"{}} to choose the linear algorithm (default option).
\end{DoxyItemize}
\item We may choose between the three types input range configuration\+:
\begin{DoxyItemize}
\item Add {\ttfamily -\/D C\+A\+SE=\char`\"{}\+A\+V\+E\+R\+A\+G\+E\char`\"{}} to generate input values randomly chosen from {\ttfamily \mbox{[}-\/100;100\mbox{]}} (default option).
\item Add {\ttfamily -\/D C\+A\+SE=\char`\"{}\+W\+O\+R\+S\+T\char`\"{}} to generate only null or negative input values.
\item Add {\ttfamily -\/D C\+A\+SE=\char`\"{}\+B\+E\+S\+T\char`\"{}} to generate only positive input values.
\end{DoxyItemize}
\end{DoxyItemize}

While compiling you may also turn on/off the following preprocessing directives\+:


\begin{DoxyItemize}
\item {\ttfamily -\/D D\+E\+B\+UG}\+: to print out messages that explain intermediate steps of the running algorithm.
\item {\ttfamily -\/D P\+R\+I\+NT}\+: to print the input and the output range.
\end{DoxyItemize}

By default, i.\+e. if no preprocessing directive is provided while compiling, the program runs with the following parameters set\+:


\begin{DoxyItemize}
\item Range with 20 elements in the average case.
\item The linear algorithm.
\item Not printing the input/output range.
\item Not displaying any debug message.
\end{DoxyItemize}

\section*{4. Running}

To run you may inform the size of the input range, which should be filled with random values\+: {\ttfamily \$ .\textbackslash{}filter \mbox{[}input\+\_\+size\mbox{]}}

For instance, the example below


\begin{DoxyCode}{0}
\DoxyCodeLine{\$ .\(\backslash\)filter}
\end{DoxyCode}


runs filter with 20 random values, whereas the example below


\begin{DoxyCode}{0}
\DoxyCodeLine{\$ .\(\backslash\)filter 100}
\end{DoxyCode}


runs filter with 100 random values.

In case you provide an invalid input value, the program defaults to 20. 